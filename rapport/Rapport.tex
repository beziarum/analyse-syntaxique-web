\documentclass[12pt]{article}

\usepackage[T1]{fontenc}
\usepackage[utf8]{inputenc}
\usepackage[francais]{babel}
\usepackage{hyperref}
\usepackage{url}
\usepackage{color}
\usepackage[usenames,dvipsnames]{xcolor}
\usepackage{verbatim}
\usepackage{tocbibind}
\usepackage{array}

\begin{document}
\title{Projet de programmation d'analyse syntaxique}
\author{Paul BEZIAU, Antoine BORDE, Martin BAZALGETTE}

\maketitle
\newpage
\tableofcontents
\newpage


\section{Présentation du projet}
Le projet d'analyse syntaxique consistait en la réalisation d'un compilateur
de site web qui retourne un code html valide à partir d'un code source "factice"
donné par l'énoncé.
\newline
L'énoncé proposait plusieurs étapes au développement permettant une avancée
pas à pas des possibilités offertes par ce code factice et de sa gestion
par le compilateur.
\newline
Le programme est composé de 3 parties différentes. l'analyseur lexical (en langage flex) ,
l'analyseur syntaxique permettant de gérer la grammaire (en langage bison) et les structures
permettant le traitement des données et leur transformation en code html (en langage C).
\newline
A chacune des étapes de dévelopement, il s'agissait de vérifier dans un premier
temps la qualité de la grammaire (si elle accepte ce qu'elle doit et refuse
le code invalide), et dans un second temps la bonne gestion de nos structures
et du traitement réalisé sur les données afin de les compiler en code html valide.

\newpage

\section{Les premieres forêts}

Notre premier objectif était d'implementer un début d'analyseur lexical et syntaxique
permettant d'accepter le premiere éxemple du code factice donné dans l'énoncé.

\subsection{Les structures}

Pour commencer ce projet, nous nous sommes penchés sur la gestion des strutures
"tree" et "attributes" données par l'énoncé. Nous avons implémenté les créateurs et les
libérateurs ainsi que les accesseurs et les modificateurs. Rien de très compliqué a ce niveau
là, la dificulté était de bien tout construire au niveau de l'analyseur syntaxique.

\subsection{L'analyseur lexical}

Nous nous sommes ensuite penchés sur l'analyseur lexical. Nous voulions essayer de
minimiser le plus possible le traitement au niveau de flex afin que tout soit géré
par notre grammaire dans l'analyseur syntaxique. Nous avons donc implémenté
quelque chose de très simple qui se contente de renvoyer des tokens pour tous les
characters ou mots que nous pensions nécessaires. Cela sans gestion d'états ou actions
particulières.

\subsection{L'analyseur syntaxique}

Nous sommes ensuite ataqués à l'analyseur syntaxique, la partie centrale de ce projet.
Nous avons défini une suite de règles plutôt simples dans un premier temps. Le but
n'etait pas encore de tester si nos arbres et nos forêts s'imbriquaient comme il fallait,
mais simplement de vérifier que nous avions une grammaire acceptante quand c'etait
le cas et de retourner une erreur le reste du temps.
\newline
Outre toutes les petites difficultés classiques dues à notre manque d'expérience
sur la gestion des grammaires, Nous avons rencontré ici notre première
grosse dificulté.

\subsubsection{La gestion des espaces}

Les espaces représentaient le premier élément compliqué à gérer. il doivent parfois étre acceptés,
d'autres fois non. Il est difficile de gérer dans la grammaire tous ces cas de figure.
Dans un premier temps, nous avons défini une regle "SPACES: SPACES | %empty", et nous
avons mis des SPACES un peu partout dans la grammaire. Mais celà entrainait 2 problèmes :
le premier était que même si notre grammaire devenait acceptante sur les bons fichiers,
elle l'était aussi à des moments où elle n'aurait pas du l'être. Le deuxième était un
nombre phénoménal de conflits reduction/réduction (+ de 50).
\newline
Nous avons donc décidé après réfléxion de nous résoudre a complexifier le code lex
afin d'éffectuer un prés traitement sur les espaces à accepter ou non.
Nous avons défini un état ainsi qu'une variable "lastIsSpace" chargée de mémoriser
si les derniers caractères etaient des espaces. Celà nous a permi de ne retourner des token
différent quand il été précéder d'un espace et ainsi d'énormement simplifier la grammaire.
Nous avons alors réussi a obtenir une grammaire satisfaisante.



\newpage

\section{Le compilateur}

Notre programme commençant à reconnaitre les premiers éléments de syntaxe
du code donné par l'énoncé, nous nous sommes lancés sur la réalisation de notre
première vraie compilation.
\newline
\newline
Mais nous avons alors étés confrontés au changement de sujet concernant les structures
et leur implementation.

\subsection{Changement total des structures}

Le sujet a donc proposé une nouvelle implémentation des structures (déjà au point) afin de nous
permettre de nous concentrer essentiellement sur l'analyse syntaxique (qui est effectivement
le veritable objectif de cette UE). Cependant il était parfois difficile de comprendre
le fonctionnement de ce code, et de bien gérer les structures pour qu'elles gérent les données
de la manière attendue. La gestion des structures restait donc un problème difficile
à gérer.

\subsection{La fonction emit}

L'implementation donnée dans le sujet n'était pas encore tout à fait complète.
Il restait notamment à implémenter la fonction emit qui se charge de recupérer
notre arborescence de structure pour réecrire les données sous la forme d'un code
html valide dans un fichier.
\newline
\newline
Une fois les nouvelles structures comprises, l'ecriture des balises n'était pas
d'une grande difficulté. La gestion des indentations et des retours à la ligne
était quand à elle plus complexe. Nous avons dans un premier temps opté pour
une fonction récursive qui intègre dans ses parametres une variable indiquant
le nombre d'indentation a éffectuer. Nous retournons également un booléen qui
nous aide à déterminer, selon les cas, si le retour à la ligne est nécessaire ou pas.
\newline
A ce stade, le résultat n'etait pas parfait mais convainquant et suffisamment clair
pour commencer les tests.


\subsection{Les premières compilations}

Dès les premières tentatives, nous avons eu de nombreuses erreurs dues à la mauvaise
gestion des structures. Nous n'avions notamment pas compris comment chainer
les mots. Nous avons décidé de les considérer comme des forêts qui s'empilent les
une aprés les autres jusqu'à ce que tous les mots de la chaine soient insérés.
Aprés quelques ajustements et corrections, nous avons obtenu un programme qui
compilait correctement le code factice dans sa version simple (gestion des noeuds,
des attributs et de l'operateur emit).

\newpage


\section{Enrichissement de la syntaxe}

La premiere partie étant bien maitrisée, nous nous sommes lancés dans l'enrichissement
de la gestion de la syntaxe du langage.

\subsection{Les variables}

Les variables n'étaient pas la partie la plus dure à gérer du point de vue de
la grammaire. Nous avons décidé de gérer indépendamment les variables situées
en début de fichier comme par exemple une suite de déclarations, et les variables locales.
Nous avons défini quelques tokens suplémentaires, ajouté quelques régles à la
grammaire et le résultat a vite était satisfaisant.
\newline
Mais la gestion de ces variables au sein des structures fut plus compliquée.
En effet, nous avançons à taton quant au fonctionnement de la machine virtuelle, et
nous mettons du temps à chaque nouvelle action qu'il faut effectuer sur ces
dernières.

\subsection{Les fonctions}


\subsection{Les expressions conditionnelles}


\newpage
\section{Conclusion}

\subsection{Etat d'avancement du projet}

Une estimation de l'avancement du projet.
\newline

\newcolumntype{M}[1]{>{\raggedright}m{#1}}
\begin{tabular}{|M{3cm}|M{3cm}|M{3cm}|}
    \hline
    objectif & grammaire & gestion de l'arborescence \tabularnewline
    \hline
    noeud et attributs &  validé & validé \tabularnewline
    \hline
    variable & validé & validé \tabularnewline
    \hline
    fonction & validé & non validé \tabularnewline
    \hline
    condition & validé & non validé \tabularnewline
    \hline
    gestion sur plusieur fichier & non validé & non validé \tabularnewline
    \hline
 \end{tabular}

\subsection{Le mot de la fin}
Nous avons trouvé que ce projet était de loin le plus dur que nous ayons eu a
faire au sein de la licence informatique. Long, difficile à comprendre et
à appréhender ainsi que complexe a réaliser. C'est la premiere fois
que nous narrivons pas au bout du sujet dans les temps impartis, et meme un
delai suplémentaire ne nous aurait pas garanti d'arriver à terme du projet.
\newline
\newline

Nous avons fait le choix, sur la fin, de nous concentrer sur la grammaire
qui semblait etre la partie la plus attendue par les enseignants.
\newline





\end{document}
\grid
